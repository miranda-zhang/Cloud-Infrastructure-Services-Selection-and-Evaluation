\chapter{Conclusion}
\label{cha:conclusion}
We conclude our contributions and point out future directions in this area.

\section{Impact}
This thesis mainly considers the problem of automatic Cloud IaaS service discovery and comparison.
It also explored ways to forecast cloud usage in order to estimate cost, thus providing optimal Cloud deployment options (i.e. which provider, total cost, what kind of servers, how many of them, the expected average network performance e.t.c).  

Although elasticity, cost benefits and abundance of resources motivate many organizations to migrate their enterprise applications to the Cloud, end-users (e.g., CIOs, scientists, developers, engineers, etc.) are faced with the complexity of choosing the right set of Cloud services for deploying their applications. Manually reading Cloud providers’ documentation to find out which services are suitable for building their Cloud-based application is a cumbersome task for decision makers. The multi-layered organization (e.g., SaaS, PaaS, and IaaS) of Cloud Services, along with their heterogeneous types and features makes the task of service identification a hard problem. For example, EC2 instances, virtual/compute/cloud servers are all the same thing; S3, Cloud files, object store all refers to storage. In addition to the actual price difference for similar service among various providers, a range of pricing models for how services are charged are also making it more complicated, like Pay as you go, Spot Instance or bidding, two part tariff, block-declining, free for a period or discounted with bulk buy.Hence for dealing with the complexity of choosing from large number of heterogeneous cloud services from diverse providers, end-users need access to specialized intermediaries which can act as a “one-stop-shop” for procuring and comparing Cloud services. 

We addressed the aforementioned problems by introducing a number of techniques with their implementation to form a Cloud recommendation tool-set which enables simplified and intuitive cloud service selection. It allows multi-criteria search on infrastructure cloud offers across different cloud providers.It is a semi-automated approach aids in the network-QoS-aware 
selection of cloud services, along with a unified domain model that
is capable of fully describing infrastructure services in cloud
computing. This approach takes into account of real-time and variable network QoS constraints,
and applies a utility function that combines multiple selection criteria
(e.g., the total cost, the maximum size limit for
storage, and the memory size for compute instance)
pertaining to storage, compute, and network services.

The Cloud recommendation tool-set is expected to be of significant value to end-users, who are presently:
\begin{enumerate}
\item considering migrating their applications to Cloud services for cost savings.
\item hard-pressed to understand the cost and benefits of moving to Cloud, especially when the market evolves so rapidly.
\end{enumerate}

\section{Contributions}
The cloud has great potential for a large variety of users with diverse needs, but the selection of a the right provider is crucial to this end. Aiming to eliminate potential bottlenecks that limit the ability of general users to take advantage of cloud computing, we present a comprehensive solution set, which allows user to make multi-criteria selection and comparison on IaaS offers considering QoS. We hope our research will drive even greater adoption of the cloud and boost the expansion of the cloud hosted applications. 

\subsection{Standardization of Terminologies Concepts and Processe in Cloud}
We have proposed ontology for classifying and representing the configuration information related to Cloud-based IaaS services including compute, storage, and network. The proposed ontology is comprehensive as it captures both static confiugrations and dynamic QoS configurations on the IaaS layer.

By proposing the CoCoOn Ontology, and implementing relevant tool,
empowering automatic cross linking data from different providers as well as external domains,
i.e. Geo Locations, Units, Business Service (GoodRelations).
Thus, unify the process of selection and comparison Cloud IaaS Services.

This work will also help readers in clearly understanding the core IaaS-level Cloud computing concepts and inter-relationship between different service types. This in turn may lead to a harmonization of research efforts and more inter-operable Cloud technologies and services at the IaaS layer.

\subsection{Network-Aware QoS Computation}
A generic service that helps in collecting network QoS values
from different points on the Internet (modeling a big data source
location) to the cloud data centers.

\subsection{Problem Formulation}
A clear formulation of the research problem by identifying the most important
cloud service selection criteria relevant to specific real-time
QoS-driven applications, selection objectives, and cloud service
alternatives.

\subsection{Multicriteria Optimization}
An analytic hierarchy process (AHP)-based decision making 
(service selection) technique that handles multiple
quantitative (i.e., numeric) and qualitative (a descriptive and
nonnumeric, such as location, CPU architecture, i.e., a 32- or
64-bit operating system) QoS criteria. The AHP determines
the relative importance of criteria to each user by conducting
pairwise comparisons.

\subsection{Resource Usage Estimation based on Benchmarking and Performance Modelling}

\subsection{Cloud Recommendation Toolset}
We proposed a declarative system (CloudRecommender) that transforms
the cloud service configuration selection from an ad-hoc process that involves
manually reading the provider documentations to a process that is structured, and to a
large extend automated. Although we believe that CloudRecommender leaves scope
for a range of enhancements, yet provides a practical approach. We have implemented
a prototype of CloudRecommender and evaluated it using an example selection
scenario. The prototype demonstrates the feasibility of the CloudRecommender
design and its practical aspects.

\section{Reflection and Future Work}
There are a number of possible extensions that can be made to CoCoOn:
\begin{enumerate}
    \item More providers should be included.
    \item Using Custom Datatypes \texttt{cdt:ucum} \cite{lefrancois_eswc_2018} to define custom units in CoCoOn, and so it can be implemented to handle conversions properly.
    \item Improve \href{https://github.com/miranda-zhang/cloud-computing-schema/tree/master/example/geonames_rdf/azure}{mapping regions to Geoname dataset}.
    \item Model various discounts.
\end{enumerate}

For the current implementation of the
CloudRecommender system, we have defined the Cloud
infrastructure layer (IaaS), providing concepts and relations
that are fundamental to the other higher-level layers.
For future work, to cover both PaaS
and SaaS layers would be a nice inclusion.
In terms of system implementation, we could migrate the infrastructure services definitions to a RDF database and use, for example, SPIN templates \cite{SPIN} to encode our procedures in SPARQL \cite{SPARQL}.
Moreover, it would also be nice to capture the dependency of services across the layers. For example, investigating concepts and relationships for identifying the dependencies between compute service (IaaS) configurations and the type of appliances (PaaS) that can be deployed over it. For example, before mapping a MySQL database appliance (PaaS) to a Amazon EC2 compute service (IaaS), one needs to consider whether they are compatible in terms of virtualization format. 
Other parameters worth including into the decision making process are SLA and legal compliance \cite{mouratidis2013framework} information.

Another avenue that we would like to explore is spot instance bidding/auction market. There are a number of
research focus on service brokerage that can be of good synergy to our work.
