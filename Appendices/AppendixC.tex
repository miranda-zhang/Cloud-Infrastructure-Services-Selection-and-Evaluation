\chapter{Publications Recap}
Revisions on old works.

\section{Investigating decision support techniques for automating Cloud service selection}
\label{paper:proposal}
This paper briefly introduces features of Cloud Infrastructure as a Service, then gives an overview on the generally faced challenges by Cloud users in the first step of picking from various options. Problems depicted are grouped under the following areas:

\begin{enumerate}
	
  % 1
  \item \label{itm:service_identification_representation}
  Automatic service identification and representation.   
  
  \begin{enumerate}[label*=\arabic*.]
  
  	% 1.1
    \item \label{itm:service_discovery} 
    How to automatically fetch service description published by Cloud providers and present them to decision makers in a human readable way?
    % 1.2
    \item \label{itm:ontology}
    Can we develop an unified and generic Cloud ontology to describe the services of any Cloud provider which exists now or may become available in the future?
  \end{enumerate}
  
  % 2
  \item \label{itm:selection_comparison}
  Optimized Cloud Service Selection and Comparison.
  
  \begin{enumerate}[label*=\arabic*.]
  	% 2.1
    \item \label{itm:comparison}
    How well does a service of a Cloud provider perform compared to the other providers? 
    % 2.2
    \item \label{itm:compatibility}
    Which Cloud services are compatible to be combined or bundled together?
    % 2.3
    \item \label{composition}
    How to optimize the process of composite Cloud service selection and bundling?
  \end{enumerate}  
  
  % 3
  \item \label{itm:interface}
  Simplified interfaces for Cloud Service Selection.
  
  \begin{enumerate}[label*=\arabic*.]
    \item How to develop interfaces that can transform low, system-level programming to easy-to-use drag and drop operations? 
    \item Will such interfaces improve and simplify the process of Cloud Service Selection and Comparison? 
  \end{enumerate} 
  
\end{enumerate}

The research required to address question \ref{itm:service_discovery} is discussed in paper \ref{paper:ontology} . One option to address the knowledge discovery problem is everyone follows a unified standard, then information can be automatically fetched, but this requires considerable efforts for Cloud providers. Otherwise, a software for harvesting information needs constant human update to accommodate every single change in Cloud providers’ websites. 

A few sub questions are identified under each area, but there is not enough time to address all of them during the course of this research. Under question \ref{itm:selection_comparison}, my research is primarily focused on solving \ref{itm:comparison}. Service composition related questions may be  investigated in future work.

\subsection{Remark}
Some  of the works reviewed in this paper are outdated, for example, Yuruware revised it's scope to focus on cloud disaster recovery now, which is quite different from our research. Similarly, some of the early days Cloud Providers no longer provide public IaaS services. For example, GoGrid was included in our initial study, but they have shifted its focus to provide consultation, despite the fact that they still claims to have data center around the globe, their business now days are mostly helping businesses to architect, deploy and manage multi platform hybrid IT solutions (i.e. in public, private and hybrid clouds). However, the non-standardized terminology problem still largely remains with other cloud providers.
Although Cloud have been around a while now, the market environment of IaaS is still changing rapidly, with new vendors (e.g. DataCentred, atlantic.net)and services (e.g. Mobile and Internet of Things related services) introduced, the corresponding concepts also evolves as a result. However, Amazon Web Service has the trend to become the de Facto Standard, at the same time, big name competitors (i.e. Hewlett-Packard, IBM) would very much like to knock Amazon o its perch. Therefor, there is still a lot of discussion around building the actual de jure Standard.

Although we didn't conduct surveys, We analyzed a lot of case studies published by the providers. It is more time efficient and easier compare to surveys, with no need to obtaining ethics approval. Those case studies are based on real life businesses scenarios, from a broad range of disciplines.

Genetic Algorithm(GA) was initially proposed because it shows some promising results for a problem another student of my supervisor is working on \cite{CloudGenius}. But after some experimentation, we realized that Genetic Algorithm(GA) is not appropriate for our problem. GA does not always find the global optima, it's
relatively time-consuming and does not scale well with complexity (i.e. search space grow
exponentially for large number of Cloud service options) \cite{GA_wiki}. But our current approach will find the global optima and is more efficient.

Deploy application in Cloud involves a lot of complex low level programming, thus making
it a time consuming and challenge task even with one Cloud platform, let along manage a
hybrid Cloud environment. Among the various attempts targeting this Cloud orchestration
problem, Cloudify \cite{Cloudify} seems to be a promising solution. Since Cloud orchestration is not the
main focus of my research, my work would concentrate only on providing related interface
for comparing Cloud Services. At the time of writing this paper, we considered integrating
the proposed Cloud Recommender with a Cloud orchestration tool as possible future
work.
 
There are some research works using AHP to solve cloud related decision problem. One
paper combined AHP with benefit-cost-opportunity-risk (BCOR) analysis to select the best
cloud computing deployment model \cite{AHP_BCOR}, but it did not consider the complexity of dealing with hundred thousands of public cloud options, it only provides a high level methodology, specifically to decide whether to choose public, private, or hybrid cloud deployment.  
Another paper \cite{CloudGenius} combined AHP and GA to automate the migration of web application clusters to public clouds. But it is looking at VM image bundles, in comparison,
CloudRecomender is comparing IaaS offers.

In Chapter \ref{cha:selection}
we employed AHP to find th optimal result considering
multiple conflicting criteria. We did not try the simple additive weighting approach,
because despite the "simple" in its name, computational wise it is not much a simpler than
AHP. Space pruning is not implemented since we decide to follow the declarative programming
paradigm, which allows us to taking advantage of the optimized query operations in SQL
(e.g. select and join). Knowing that modern databases often implement those operation
very efficiently, it is not within our scope to improve it further, instead we tried cache
query results for repeated query and pre-compute some values to reduce computation time.

Although there is not a full satisfaction study, we made numerous attempts to 
get evaluations from research community and industry. We have presented our system 
at various conferences (CloudCom, GECON), competitions
(make into the finalist of NASSCOM Inovation Student Awards) and to our research
partners (Institute of Remote Sensing and Digital Earth, Chinese Academy of Sciences add
Myriads research team, INRIA). We have modified our system according to their
feedbacks. 

\subsection{Attribution statement}
The candidate undertook the research and analysis, and wrote the chapter. 
Dr Ranjan provided valuable guidance. Dr Haller provided insights on semantic modeling. 
Dr Georgakopoulos inputs ideas on experimental result validation. 
Dr Strazdins provided reflections and comments through out the writing of this paper.

\section{An Ontology based System for Cloud Infrastructure Services Discovery}
\label{paper:ontology}
This paper explores ontology based methods to address the research question identified 
in section \ref{subsec:service_discovery}. 

An ontology was developed using semantic Web languages like the Resource Description Framework (RDF) and the Web Ontology Language (OWL), thus answered question \ref{itm:ontology}.
The semantic web initiative is proposed to simplify the information discovery process on the web, to take advantage of existing effort, an OWL-based ontology model have been developed as an attempt to address the previously mentioned knowledge discovery and data standardization problem. In particular, the Cloud Computing Ontology (CoCoOn) defines functional and non-functional concepts, attributes and relations of infrastructure services. This ontology facilitates the description of Cloud infrastructure services; and through mappings from provider descriptions, facilitates the discovery of infrastructure services based on their functionality.
By modeling of service descriptions published by Cloud providers according to the developed ontology, this paper validates the expressiveness of ontology against the most commonly available infrastructure services including Amazon, Microsoft Azure etc.

\subsection{Attribution statement}
The candidate undertook most of the conception design and analysis. 
Dr Ranjan provided valuable guidance. 
Dr Haller added expertise in Web Ontology Language, and led the implementation of CoCoOn.
Dr GeorgakopoulosI and Dr Nepal provided reflections and comments through out the writing of this paper.
Dr Menzel provided useful suggestions by sharing experience on design of decision support system for web server cloud migration.
  
\section{Discovery-Driven Service Oriented IoT Architecture}
\label{paper:soa_iot}

This paper discusses the problem of service and data discovery in the context of Internet of Things. This resembles the service discovery problem discussed in section \ref{subsec:service_discovery}.

Cloud computing has proved to be the de facto standard for delivering internet-based application services in particular supporting IoT applications and services. Such applications will require dynamic mash-ups of things and services to be created across the IoT layers. This leads to the most important and demanding challenge i.e. how to interoperate and integrate data to suit the application needs autonomously? Just like how the human brain can perceive and integrate things seamlessly to infer contextual insights, the question that will govern the IoT ecosystem is, what and how will IoT deliver to me?

This paper presents a vision of a future IoT system architecture that is driven by service discovery across every layer of the IoT system. This paper identified the  gap in current IoT architectures, that is disconnected efforts focusing on specific layers of the IoT stack, but missing the integrate service oriented concepts across the layers allowing autonomous composition of IoT applications. The work presented in this paper addresses these problem by proposing a blue print architecture for a discovery driven service oriented IoT architecture.
This paper also presents the challenges in discovery, description and representation of service and integration. The proposed blueprint architecture aims to address these challenges by embedding of discovery and integration of services at each of the devices, data and applications layers of the abstract IoT model. 
Finally, this paper provided discussions into how the proposed model can be realized by taking into consideration of some previous work in the areas of device discovery in OpenIoT and cloud discovery in Cloud Recommender.

\subsection{Attribution statement}
This is a joint paper that the candidate was involved, but the work primarily was conducted by the leading authors.

\section{City Data Fusion: Sensor Data Fusion in the Internet of Things}
\label{paper:ct_data_fusion}
This paper addresses major open research issues related to sensor data fusion, likewise to paper paper \ref{paper:soa_iot}, some of the problems touched by this paper are related to research question in section \ref{subsec:service_discovery}.  

This paper highlights the importance of sensor data fusion in IoT application such as smart
cities applications. Mainly, data fusion operations can be applied at two levels: cloud level and within the network level. Cloud level devices have access to unlimited resources and hence has the capability to apply complex data mining algorithms over the data generated by large number of lower level sensors. After understanding the environment, the cloud can generate actions that need to be taken appropriately. In-network sensor data fusion is important to reduce the data transmission cost. However, low-level nodes may not have the full view of the environment. In such situation, a middle-ware such as CloudRecommender would be needed to enable complex cooperative operations.

\subsection{Attribution statement}
This is a joint paper that the candidate was involved, but the work primarily was conducted by the leading authors.

\subsection{A Declarative Recommender System for Cloud Infrastructure Services Selection}
\label{paper:single_criteria}

This paper addresses question \ref{itm:service_identification_representation} and \ref{itm:interface}. It explains the implementation of CloudRecommender, a decision support system based on our ontological model for the selection of infrastructure Cloud service configurations. The benefits to users of CloudRecommender include, for example, the ability to estimate costs, compute cost savings across multiple providers with possible trade offs.

Though branded calculators are available from individual cloud providers, such as Amazon, Azure and GoGrid, for calculating service leasing cost, it is not easy for users to generalize their requirements to fit different service offers (with various quota and limitations) let alone computing and comparing costs, thus our system is needed.

The ontology CoCoOn proposed in paper \ref{paper:ontology} was implemented as a relational model. It is then populated with data collected from various Cloud providers' websites. We also profiled the corresponding QoS statistics of those services. By extending CoCoOn into a decision support system, with reasonable interface, we partially addressed question \ref{itm:selection_comparison} and \ref{itm:interface} .

CloudRecommender formally captures the domain knowledge of services using a declarative logic-based language, so procedures in CloudRecommender are transactional and well defined in SQL semantics. The CloudRecommender system also leverages the Web-based widget programming technique that transforms drag and drop operations to low-level SQL transactions. It provides an user-friendly service interface that maps user requirements to available infrastructure services.

\subsection{Remark}
This initial prototype did not apply optimization techniques, single objective search was implemented.

\subsection{Attribution statement}
The candidate undertook the majority of the research work, from design, implementation, data collection to analysis and wrote the chapter. 
Dr Ranjan provided valuable guidance. 
Dr Nepal and Dr Menzel provided advice on all sections.
Dr Haller added insights on how to implement OWL-ontology as relational model.

\section{Investigating Techniques for Automating the Selection of Cloud Infrastructure Services}

Similar to paper \ref{paper:single_criteria}, the work of this paper primarily addresses research problem \ref{itm:service_identification_representation} and \ref{itm:interface}, it further illustrates the detailed design of CloudRecommender after the latest improvements.

This paper gives an overview of previous work, which will help readers in clearly understanding the core IaaS-level Cloud computing concepts and inter-relationship between different service types.

This paper also revised the proposed ontology for classifying and representing the configuration information related to Cloud-based IaaS services. The corresponding  implementation is updated, it further expose a RESTful (REpresentational State Transfer) APIs (application programming interface) that help external applications to programmatically compose infrastructure cloud services based on the CloudRecommender selection process.

\subsection{Attribution statement}
This paper is an extended journal version of a previous conference paper.
The candidate undertook the work for extending the original system 
(including both model design and implementation) and wrote about the new work.
Dr Ranjan provided valuable guidance.
The other co-authors assisted in writing.

\section{An Infrastructure Service Recommendation System for Cloud Applications with Real-time QoS Requirement Constraints}
This paper addresses research problem \ref{subsec:service_comparison} . 

In order to select the best mix of service offering from an abundance of possibilities, application owners must simultaneously consider and optimize complex dependencies and heterogeneous sets of criteria (price, features, location, etc.). 
For instance, it's not enough to just select optimal cloud storage service, corresponding computing capabilities (at where data is located, so no additional data transfer is needed) are essential to guarantee that one is able to process the data as fast as possible while minimizing the cost. 
Matching results to decision makers’ requirements may also involves bundling of multiple related Cloud services, computing combined cost (under different billing models and discount offers), considering all possible (or only valuable) alternatives and multiple selection criteria.

To this end, we present multi-criteria decision making technique that builds over well known AHP method. AHP handles multiple quantitative (i.e. numeric) as well as qualitative (descriptive, non numeric, like location, CPU architecture: 32 or 64 bit, operating system) criteria. It determines the relative importance of criteria to each user by conducting pair-wise comparisons. The proposed technique is applicable to selecting Infrastructure as a Service (IaaS) cloud offers, and it allows users to define multiple design-time constraints. These requirements are then matched against our knowledge base to compute possible best fit combinations of cloud services at IaaS layer. 

A number of research and commercial projects provide simple cost calculation or benchmarking and status monitoring, but none is capable to consolidate all aspects and provide a comprehensive ranking of infrastructure services. For instance, CloudHarmony \cite{cloudharmony_speedtest} provides benchmark results without considering cost, Cloudorado \cite{Cloudorado} calculates the price of IaaS-level CPU services based on static features (e.g., processor type, processor speed, I/O capacity, etc.) while ignoring dynamic QoS features (e.g. latency, throughput etc.). Another project is the Swinburne University's Smart Cloud Broker Service \cite{SwinburneCloudBroker}. From the screen cast they released we can tell that their benchmarking is done in real-time, which means users have to wait for the results to come back. We have considered this kind of situations, but decided to collect the benchmarking result beforehand. Because this way no matter how many cloud providers users want to compare against, they can still get the result with minimum (or no) waiting time. Another reason we choose to do it this way is because, at any particular point in time, the network benchmark result is not conclusive as performance fluctuates during time, so we use aggregated average which is a more reliable overall indication.

There are two distinctive features the proposed solution provide during ranking and comparing various vendor services: 
\begin{enumerate}
	\item
	Allow users to choose to include the QoS requirements during comparison.
	\item
    When users want to take into account mixed qualitative (e.g. hosting region, operating system type) and quantitative criteria, we apply the Analytic Hierarchy Process (AHP) to aggregate numerical measurements and non numerical evaluation. Results are personalized according to each user's preferences, because AHP takes users' perceived relative importance of criteria (pair-wise comparisons) as inputs.
\end{enumerate}

\subsection{Attribution statement}
The candidate led the research work, undertook most of the problem formulation, data analysis, implementation, experimentation and writing.
Dr Ranjan provided valuable guidance on all aspects. 
Dr Menzel added expertise in AHP, and helped with the implementation.
Dr Nepal and Dr Strazdins provided advice on conception design and architecture.
Dr Jie and Dr Wang assisted with formulation.

\section{Early Observations on Performance of Google Compute Engine for Scientific Computing}

This paper’s evaluation work on Google Compute Engine (GCE) is relevant to research problem \ref{subsec:service_comparison}.

In 2013, public Cloud services was still new for most, and it was regarded as a potential and encouraging paradigm for scientific computing. Back then, Google Compute Engine (GCE) was just available, and claimed to support high-performance and computationally intensive tasks, while little evaluation studies can be found to reveal GCE’s scientific capabilities. Considering that fundamental performance benchmarking is the strategy of early-stage evaluation of new Cloud services, we followed the Cloud Evaluation Experiment Methodology (CEEM) to benchmark GCE and also compare it with Amazon EC2, to help understand the elementary capability of GCE for dealing with scientific problems. 

The experimental results and analyses show both potential advantages of, and possible threats to applying GCE to scientific computing. Following the same evaluation methodology, different evaluators can replicate and/or supplement this fundamental evaluation of GCE. Based on the fundamental evaluation results, suitable GCE environments can be further established for case studies of solving real science problems.

The potential advantages of using GCE: Relatively high memory data throughput. Different GCE types seem to have consistent type of virtual memory. Relatively fast storage transaction speed.
Relatively fast computation transaction speed with single process. Relatively computationally economic, given generally high computation performance/price ratio.

The possible threats to using GCE: Relatively low communication data throughput between
US and European data centers. Distributed computing crossing both data centers may need to be well balanced. Considerable variability of storage performance. GCE seems to have a lack of dedicated storage cache. Considerable variability of computation transaction
speed on high memory VM type. Relatively poor (original) scalability when switching
process numbers on individual VMs. GCE’s hyperthread-based virtual core seems not a suitable mechanism for parallel computing.

\subsection{Attribution statement}
The candidate helped with the experiments, including setting up the environment on both AWS and GCE, data collection of repeated tests.

\section{An Overview of Cloud Based Content Delivery Networks: Research Dimensions and State-of-the-Art}

This paper written in 2015, presents a state-of-the-art survey on current commercial and research/academic Cloud-based Content distribution networks (CCDNs). Some of the existing challenge provides motivation for Cloud recommendation service.

Cloud CDN providers are mostly based on one cloud platform and lacks support for the emerging form of content distribution namely dynamic user-generated content. Since, the solutions are based on a single cloud providers, the services lack consideration for cost models when taking advantage of cloud content storage spanning multiple cloud providers. Further, most solutions lack support for penalization at user level. 

The future CCDN should be based around the need to support user created content and demonstrates ability to support hybrid cloud platforms and addressing the challenges such as QoS, SLA, costing introduced by hybrid clouds.

\subsection{Attribution statement}

\section{Service recommendation based on Benchmarking and Analytical Performance Model}
\subsection{Attribution statement}